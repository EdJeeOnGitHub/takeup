\documentclass{article}
\title{Heckman Lifecycle Notes}
\usepackage[margin=0.5in]{geometry}
\usepackage{bm}
\usepackage{amsmath}
\usepackage{graphicx}
\usepackage{geometry}
\usepackage{xcolor}
 \geometry{
 a4paper,
 total={170mm,257mm},
 left=20mm,
 top=20mm,
 }


\begin{document}
\maketitle



\section*{Some Potential Action Points}
(There were lots of random things raised.)
\begin{itemize}
    \item Add back blood donation slide.
    \item Selection into who's getting dewormed
    \item JH: Look at family effect - do families go together or do singletons go?
    \item MK: Using individual level distance?
    \item Change notation on $\lambda$ slide so it doesn't look like a random effect. 
\end{itemize}


\section*package{Questions in Chronological Order}
\textcolor{blue}{Blue denotes Ed's editorialising.}
\newline
\newline
\textbf{JH}:\emph{Slide 5} What are beliefs about praise/stigma?

\emph{
    \textcolor{blue}{
    Maybe we should have gone straight to praise/stigma - since he got so held up on 
    this? Even later in the talk he asks if people see deworming as a good thing (around the 2 hour mark.)}
}
\newline
\newline

\textbf{JH}: This is a community condition - you're not necessarily stigmatising 
yourself, but rather saying your community is dirty.

\emph{\textbf{AK}: I'm going to show data on individual level views about deworming}
\newline 
\newline

\textbf{Jin:}\emph{Slide 6} How we know unique equilibrium? Each community has 
its own idiosyncratic equilibrium takeup.

\emph{
    \textcolor{blue}{
Point one, super minor quibble, I think we veered towards making the same mistake Anna Vitali did here. We said something 
like: We're going to use our structural estimation technique and make sure things aren't too 
in the tails.
I think it'd be better if we make it clear the unique equilibrium comes from theory.
Theory gives unique equilibrium $\rightarrow$ our model is identified $\rightarrow$ we estimate parameters.
We don't want to confuse identification and estimation. In reality, they just bound 
$1 + \mu \Delta'$ in the Laws and Norms paper, so what you said is basically true but 
the identification police might get angry.}
Point two, \textbf{AK:} this is exactly why we need to introduce exogenous shocks to takeup 
so that we're not doing inference with endogenous takeup equilibrium.
}
\newline
\newline


\textbf{MK}: \emph{Slide $\approx$ 9} Do you have data on who's showing up? Are we getting 
infected/non-infected people? Do we know where people are coming from/what are their occupations? Farmers, etc.?
\emph{
    \textbf{AK}: We didn't collect this sort of data but we know from other medical trials 
    that it's typically older men who are most infected.
    We know that our treatment has no differential effect by age.
}
\newline
\newline
\textbf{MK}: \emph{Slide $\approx$ 9} Could look at lake distance due to different types of worms - I can't spell 
the worm and Anne can't say this type of worm but Michael can.
\newline
\newline
\textbf{Q}: \emph{Map slide/close-far} Are we putting PoTs in an urban area, individuals living 
in rural areas travel to the urban town to get dewormed and also do shopping - individuals are 
just signalling their wealth/ability to shop?
\emph{
    \textbf{AK}: No, we're chosing where the PoTs are. They're like grains of sand on the beach, etc.
}
\newline
\newline
\textbf{Q}: \emph{This was Bill and Melinda Gates guy I think} Since worms are endemic, are there communities where rep returns are higher/lower 
and exploit that exposure.
\newline
\newline
\textbf{JH}: How often do you have to deworm? How do people get infected? 
\emph{
    \textbf{AK}: 6 months/if you want to eradicate worms, everyone just needs to take it intensively and 
    we'd eradicate them. Random discussion across the room here.
}
\newline
\newline
\textbf{Q (Jin?)}: Why not just give to kids using current programmes and have them 
pass on to parents?
\newline
\newline
\textbf{JH}: How much does $v^*$ depend on previous deworming behaviour? Has it 
been endorsed by some local group? [Obama line]. This study has been 20/30 years now, 
are there areas where this is seen as good practice but others where it's seen as 
very remote.
Sophisticated places might be really praising deworming but less sophisticated might 
not stigmatise. Heckman really thinking about previous exposure influencing $v^*$.
\emph{\textbf{MK}: Michael deworming was probably introduced at exact same time in our study 
area but could get data on that and check. \textbf{JH}: There's been stuff going on 
in this field for years, public info to everyone, how is prior exposure influencing 
treatment effect.
\textbf{AK}: I don't find meaningful heterogeneity in my other work. 
\textbf{MK}: Our treatment is actually only soil transmitted Helminths, so 
those close to the Lake might be more less salient since different timetables for 
how often you have to deworm.
}
\newline
\newline
\textbf{Juanna}: There might be heterogeneity by awareness of externalities, stigma 
will vary with awareness of externalities. Those more aware of externalities should 
stigmatise not getting the action more. Can externalities be included in $\Delta(v^*)$ model?
\emph{
    \textbf{AK}: If there's more awareness of externalities we can think of it as 
    just amplifying social image concerns. Just driving up $\mu$. 
    Externalities aren't in the model, we could add a term just controlling for 
    average level of takeup - there'd be no interaction. We would just linearly add 
    an average takeup term.
    \textbf{MK}: Our current knowledge of externality interactions is that empirically, 
    we think these things are linear. We can't reject linearity, we don't have enough 
    power to in other work. It's not clear if there should be a presumption on 
    externalities one way or the other.
}
\newline
\newline
\textbf{Jin}: Making a point about making sure we don't divide by $0$, restricting the 
bounds of support for the social multiplier, $\frac{1}{1 + \mu \Delta'(v^*)}$.
\emph{
    \textcolor{blue}{
    Here we could have said: $\mu > 0, \Delta' > 0$ in all the support of our data so the social 
    multiplier is always well defined.}
}
\newline
\newline
\textbf{JH}: You're talking about individuals but presumably there must be a strong 
family effect. One person in the family could re-infect everyone. Do families enrol 
in the programme or just individuals?
\emph{
    \textbf{AK}: Everyone can enrol. \textbf{JH}: But do families go together or 
    do people go separately, what's the family structure? Returns as a single 
    individual vs a family, it's very different.
}
\newline
\newline
\textbf{Q}: Would people know if they were infected afterwards? Would you see it 
in your stool?
\emph{\textbf{MK}: You could survey people after and see if higher worm load people 
came before or after. [Tangent] Worm loads are really falling in Kenya.}
\newline
\newline
\textbf{Q}: You're assigned to far/close, but you're always assigned to the closest. The 
far people, are they more remote? Who employs CHVs? Can't people just give their 
bracelet to someone else?
\newline
\newline
\textbf{MK}: Do you have individual HH level distance? We could use the variation 
from individual level HH to exploit more variation? For some individuals, 
even in close/far we'll have variation in which HHs are closest.
\newline
\newline
\textbf{HL:} Since individuals can go to pharmacies, when we measure the distance 
they have to walk it's not quite the most precise measure. We're ignoring the closest 
PoT could be a pharmacy.
\emph{\textbf{AK}: But pharmacies you have to pay}
\newline
\newline
\textbf{MK}: If there's other stories going on, how would that influence 
estimation of the model?
\emph{\textcolor{blue}{This is a good point. We have the $u$ shock, so in some sense we load any 
other factor into there. If our functional form is really off, loading everything into 
$u$ isn't going to buy us much. This is just one of the weaknesses of structural 
models, we don't really have any guarantees in the case of serious misspecification. 
What we can say (in addition to the $u$-shock) is that we've used a lot of different 
functional forms for $\mu$ and the results are fairly robust which is a good sign. Can't 
say much beyond that I think. We could run alternative models for social learning etc and 
horserace but that's a whole different paper I think.}}
\newline
\newline
\textbf{Jin}: How do we know type isn't correlated with distance? 
\emph{\textbf{AK\&MK}: We randomise the sites.}
\newline
\newline
\textbf{JH}: Asking if we can identify the variance?
\emph{\textcolor{blue}{We can, fuck yeah. Then he said something about it influencing 
other terms separately etc. which I don't understand}}
\newline
\newline
\textbf{JH}: It sounds like this (WTP) experiment is really flirting with the endowment 
effect. We give a calendar, then ask them to surrender it etc.
\emph{
    \textbf{AK}: Yeah we could add a $\kappa$ term. If there is an endowment effect 
    we would have to divide by 2. But really this isn't a big deal because we're 
    assuming they're the same.
    \textcolor{blue}{I think we should avoid saying we assume they're the same, and phrase it as:
    Look, we estimate a very small monetary difference for the two, and on top of that 
    we translate monetary difference into utils very, very conservatively. This means 
    our posteriors for the private benefit actually end up being the same in this instance. 
    }
}
\newline
\newline
\textbf{JH}: How exactly are you identifying this $\mu$ in the structural model?
\emph{\textbf{AK}: We use the beliefs data!}
\textbf{JH}:  Why do I need heterogeneity on $\lambda$?
\emph{
    \textcolor{blue}{
        Our notation was confusing here, we have $\lambda \sim N(0, \tau^\lambda)$ which 
        makes it look like some sort of random effect. Actually, we have no heterogeneity on 
        $\lambda$, that was just the prior. We should change that to make it clearer.
    } 
}
\newline
\newline
\textbf{JH \& team}: Villages assigned ``Far" are remote areas discussion II. 
\newline
\newline
\textbf{JH}: Do we use individual level data that says: Have I seen somebody get the 
bracelet. And then what do I think of that person etc. Did we think about asking: How did 
you know if someone got wormed/dewormed? 
\emph{\textbf{AK}: Using the baseline data on observability - i.e. people know what 
bracelet is etc.}
\newline
\newline
\textbf{JH}: We could have just created visibility through newspapers etc. and it 
would have been the same treatment/effect?
\emph{\textbf{AK}: Well no, because you wouldn't be able to make individual type inferences. 
People have a choice to signal and we're giving people a change to signal them at all.} 
\newline
\newline
\textbf{Q}: Did we look at the difference between: How many people do you think 
have been dewormed? Was this person dewormed? How did you know they were dewormed? Vs did you 
observe one of these signals? We need to get at this relative gap. We need a multiplier 
that's relative to how many people did we see. That is, I saw this/think it's X but actually 
now that I've seen this bracelet I update and think there's actually five more people 
that actually got it - he thinks the model helps us a little bit but doesn't exactly get 
at what we want.
\textbf{JH}: This is an information story, not a reputation story. A repuation 
story would be: How many people looked at the ink on my arm.
\emph{\textbf{AK}: But why would information just change my decision to get dewormed, if I 
learnt about someone else?\textbf{JH}: Well, because I get praised. \textbf{AK}: Well, 
exactly. \textbf{JH}: There seems to be a different issue between information asymmetry vs 
just displaying information. I do this because I get exultation, other people see me. \textbf{AK}: Yes, 
that's the channel we're interested in. \textbf{JH}: But that's reputation, that's pure 
reputation. That's not information. Information and reputation are two different things. \textbf{AK}: 
They are two different things, but you can only get reputational returns if there is information or 
visibility about others actions. \textbf{JH}: I could have very strong reputation in my village, but then nobody goes.
\textcolor{blue}{I think here we should have said - exactly, you'd have a high} $\Delta(v^*)$ \textcolor{blue}{but 
}$\lambda$ \textcolor{blue}{would be very low - nobody goes because no-one cares about the reputational return. But 
we're estimating these and we don't find individuals place zero value on rep returns.} 
\textbf{JH}: But I still feel good because I did this good act \textbf{AK:} Speaking 
about self-image. We didn't really make progress on bridging this lol???
}.
\newline
\newline
\textbf{HL}: Pharmacy optimal allocation/counterfactuals point again.
















\end{document}

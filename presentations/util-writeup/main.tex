\documentclass{article}

% Packages for mathematics
\usepackage{amsmath}
\usepackage{amssymb}
\usepackage{amsthm}

% Define theorem environments
\newtheorem{theorem}{Theorem}
\newtheorem{lemma}[theorem]{Lemma}
\newtheorem{corollary}[theorem]{Corollary}
\newtheorem{proposition}[theorem]{Proposition}

\begin{document}

\title{Title of the Document}
\author{Author's Name}
\date{\today}

\maketitle

\begin{abstract}
This is the abstract of the document.
\end{abstract}

\section{Welfare}
We have utility function:
\begin{align*}
    U_i &= (w_i + b - \delta d) a_i + \mu(d) E_{-i}\left[v_i | a_i\right] \\
    &= \begin{cases}
        (w_i + b- \delta d) + \mu E \left[v_i | 1\right] & \text{if } a_i = 1 \\
        \mu E \left[v | 0\right] & \text{if } a_i = 0
    \end{cases} \\
    &= \max \left\{w_i + b - \delta d + \mu E\left[v | 1\right],
    \mu E\left[v | 0\right]
      \right\} 
\end{align*}

Therefore welfare, ignoring $\lambda$ for now:
\begin{align*}
    W(d) &= \bar{U}(d) + (1 - \lambda) \delta d \overline{a} \\
    &= \int_{-\infty}^{\infty} \max \left\{w_i + b - \lambda \delta d + \mu E\left[v | 1\right], \mu E\left[v | 0\right] \right\} f(w) dw \\ 
    &= \int^{\infty}_{w^*(d)} (w_i + b - \lambda \delta d + \mu E\left[v | w > w^*(d)\right]) f(w) dw + \int^{w^*(d)}_{-\infty} \mu E\left[v | w < w^*(d)\right] f(w) dw \\
\end{align*}
Where the last line follows from the fact we know individuals will only take action if $w_i > w^*(d)$ i.e. above cut-off type.
We use the fact: $(1 - F(w^*)) \mathcal{M}^+(w^*) + F(w^*)\mathcal{M}^-(w^*) = \overline{w}$ and take the 
reputational returns out of the first integral to get:
\begin{align*}
    W(d) &= \int^{\infty}_{w^*(d)} (w_i + b - \lambda \delta d) f(w) dw +  \mu \overline{w} \\
\end{align*}
So to maximise welfare the social planner differentiates the above and finds $d^*$ to 
solve the FOC:
\begin{align*}
    - \frac{\delta - \mu' \Delta(w^*(d))}{1 + \mu \Delta'(w^*(d))} (w^*(d) + b - \lambda \delta d) - 
    \lambda \delta \frac{\left[1 - F(w^*(d))\right]}{f(w^*(d))} &= 0
\end{align*}


\section{Takeup}
The decision to deworm is given by:
\begin{align*}
    Y_i &= \mathbb{I}\left\{w_i + b - \delta d + \mu \Delta(w^*(d)) > 0 \right\} \\
\end{align*}
So:
\begin{align*}
    \overline{Y}(d) &= \int \mathbb{I}\left\{w + b - \delta d + \mu \Delta(w^*(d)) \right\} f(w) dw \\
    &=  \int^{w^*(d)}_{-\infty} 0 \times f(w) dw + \int^{\infty}_{w^*(d)} 1 \times f(w) dw \\
    &= 1 - F(w^*(d)) 
\end{align*}
To maximise takeup:
\begin{align*}
    \tilde{d} &= \max_d \left\{1 - F(w^*(d)) \right\} \\
  0 &=    -f_w(w^*(d)) \underbrace{\frac{\partial w^*(d)}{\partial d}}_{\textrm{Social Multiplier}}  \\
\end{align*}
Whatever sets the social multiplier to zero maximises the deworming level. For 
B\'enabou and Tirole this would just set $\tilde{d} = 0$ as this would obviously max 
takeup. However, in our model:
\begin{align*}
    SM &= \frac{\delta - \mu'(\tilde{d}) \Delta(w^*(\tilde{d}))}{1 + \mu(\tilde{d}) \Delta'(w^*(\tilde{d}))}  \\
    \implies & \delta = \mu'(\tilde{d}) \Delta(w^*(\tilde{d})) 
\end{align*}
Gives the optimal distance.

TODO look at when two cooincide.

The intuition on this is quite nice. 


Look at the welfare FOC, this is the net social marginal benefit of reducing distance, $d$, by one unit: 
\begin{align*}
- \frac{\delta - \mu' \Delta(w^*(d))}{1 + \mu \Delta'(w^*(d))} \underbrace{(w^*(d) + b - \lambda \delta d)}_{\gamma}f(w^*(d))
\end{align*}

This almost coincides with the takeup maximising FOC:
\begin{align*}
 -f_w(w^*(d)) \underbrace{\frac{\partial w^*(d)}{\partial d}}_{\frac{\delta - \mu' \Delta(w^*(d))}{1 + \mu \Delta'(w^*(d))}} 
\end{align*}
Except it ignores the utility of individuals: $(w^*(d) + b - \lambda \delta d)$. 

When we maximise welfare we have to equate net social marginal benefit to the \emph{deadweight loss} of paying the 
extra subsidy (reducing distance) to all the inframarginal people - those who would have got dewormed 
anyway before we reduced distance.

When do the two coincide? When $\lambda = 0$ - there's no deadweight loss from the inframarginal 
agents and $\gamma = 1 \implies w^*(d) = -b$. In general this isn't going to happen.

\section{Main Results}

\begin{theorem}
This is a theorem.
\end{theorem}

\begin{proof}
This is the proof of the theorem.
\end{proof}

\section{Conclusion}
This is the conclusion section.

\end{document}